\section{Conclusion and Future Work}
\label{sec_conclusion}

In this paper we discussed the design and implementation of an FPGA-based approximate probability neural network.
The proposed system architecture is highly scalable and is capable of implementing other neural networks also.
The hardware approximations used enabled implementing larger networks within the resource constraints of an FPGA while keeping the classification error within tolerable limit.
These approximations also improved the maximum clock frequency of the overall design.
The automation scripts enable quickly building a hardware based neural network without the HDL design expertise.
The proposed system is available to researchers as an open-source platform~\cite{blanked}.

In the future we will be adding more library support to the platform such as multi-layer perceptrons, approximate sigmoid neurons based ANNs, long short term memory networks (LSTMs), hierarchical temporal memory networks (HTMs) etc.
Another research direction is exploiting FPGA partial reconfiguration (PR) to further improve the resource availability.
Using PR it will be possible to dynamically load and unload circuits the FPGA while preserving the communication link to the host machine.
Thus for multi-layer neural networks, it will be possible to time multiplex FPGA resources to implement different layers, thus increasing the effective logic capacity of each layer.

